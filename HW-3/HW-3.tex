\documentclass{article}
\author{Parham Alvani}
\title{AP Homework 3}
\begin{document}
\begin{titlepage}
\begin{center}
\emph{In The Name of God}
\end{center}
\newpage
\maketitle
\begin{center}
powered by \LaTeX
\end{center}
\end{titlepage}
\tableofcontents
\newpage
\section{Problem 1}
\textit{Iterator} is used to iterate through a collection and can remove elements from the
collection during the iteration.
\newline
Java classes \textit{ArrayList} and \textit{Vector} provide the capabilities of array-like data structures
that can resize themselves dynamically.
\newline
If you do not specify a capacity increment, the system will increase the size of the Vector each
time additional capacity is needed by \textit{75\%}.
\section{Problem 2}
1. Values of primitive types \textit{can not} be stored directly in a collection. \textbf{False}
\newline
2. A Set can \textit{not} contain duplicate values. \textbf{False}
\newline
3. A Map can contain duplicate \textit{values}. \textbf{False}
\newline
4. Queue doesn't have any default implementations. \textbf{True}
\section{Problem 3}
\textsc{BlueJ project has been included}
\section{Problem 4}
\paragraph{1}
API : An application programming interface (API) specifies how some software components should interact with each other.
In addition to accessing databases or computer hardware, such as hard disk drives or video cards, an API can be used to ease the work of programming graphical user interface components. In practice, many times an API comes in the form of a library that includes specifications for routines, data structures, object classes, and variables. In some other cases, notably for SOAP and REST services, an API comes as just a specification of remote calls exposed to the API consumers.
\paragraph{2}
Load Factor : A critical statistic for a hash table is called the load factor. This is simply the number of entries divided by the number of buckets, that is, n/k where n is the number of entries and k is the number of buckets.
\paragraph{3}
Hash Collision : In computer science, a collision or clash is a situation that occurs when two distinct pieces of data have the same hash value, checksum, fingerprint, or cryptographic digest.
\paragraph{4}
space/time trade-off in hashing : In computer science, a space–time or time–memory tradeoff is a situation where the memory use can be reduced at the cost of slower program execution (and, conversely, the computation time can be reduced at the cost of increased memory use). The load factor in hash table is a classic example of a memory-space/execution-time trade-off:By increasing the load factor , we get better memory utilization , but the program runs slower due to increased hashing collisions. By decreasing the load factor , we get better program speed, because of reduced hashing collisions, but we get power memory utilization, because a larger portion of the hash table remains empty.
\paragraph{5}
AutoBoxing and Unboxing : Autoboxing is the term for getting a reference type out of a value type just through type conversion (either implicit or explicit). The compiler automatically supplies the extra source code which creates the object.
\newline
Unboxing refers to getting the value which is associated to a given object, just through type conversion (either implicit or explicit). The compiler automatically supplies the extra source code which retrieves the value out of that object, either by invoking some method on that object, or by other means.
\paragraph{6}
Wrappers : A primitive wrapper class in the Java and ActionScript programming languages is one of eight classes provided in the java.lang package to provide object methods for the eight primitive types. All of the primitive wrapper classes in Java are immutable. J2SE 5.0 introduced autoboxing of primitive types into their wrapper object, and automatic unboxing of the wrapper objects into their primitive value—the implicit conversion between the wrapper objects and primitive values.
\paragraph{7}
Interface : In object-oriented programming, a protocol or interface is a common means for unrelated objects to communicate with each other. These are definitions of methods and values which the objects agree upon in order to cooperate.
\section{Problem 5}
\textsc{NetBeans project has been included}
\section{Problem 6}
\textsc{NetBeans project has been included}
\section{Problem 7}
\textsc{NetBeans project has been included}
\end{document}