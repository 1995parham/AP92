\documentclass[•]{article}
\author{Parham Alvani}
\title{AP Homework 2}
\begin{document}
\begin{titlepage}
	\begin{center}
		\emph{In The Name of God}
	\end{center}
	\maketitle
	\begin{center}
		powered by \LaTeX
	\end{center}
\end{titlepage}
\tableofcontents
\newpage
\section{Problem 1}
Each class declaration that begins with keyword \textit{public} must be stored in a file that
has exactly the same name as the class and ends with the .java file-name extension. 
\newline
Keyword \textit{new} requests memory from the system to store an object, then calls the
corresponding class's constructor to initialize the object. 
\section{Problem 2}
1. By convention, method names begin with an \textit{lowercase} first letter, and all subsequent
words in the name begin with a capital first letter. \textbf{False}
\newline
2. An import declaration is not required when one class in a package uses another in the
same package. \textbf{True}
\newline
3. Empty parentheses following a method name in a method declaration indicate that the
method does not require any parameters to perform its task. \textbf{True}
\newline
4. Variables declared in the body of a particular method are \textit{not} known as instance variables 
and can \textit{not} be used in all methods of the class. \textbf{False}
\newline
5. Floating-point values that appear in source code are known as floating-point literals and
are type \textit{double} by default. \textbf{False}
\newline
6. Java is a platform-independent language. \textbf{True} 
\section{Problem 3}
\textsc{NetBeans project has been included} 
\section{Problem 4}
\paragraph{1}
API : An application programming interface (API) specifies how some software components should interact with each other.
In addition to accessing databases or computer hardware, such as hard disk drives or video cards, an API can be used to ease the work of programming graphical user interface components. In practice, many times an API comes in the form of a library that includes specifications for routines, data structures, object classes, and variables. In some other cases, notably for SOAP and REST services, an API comes as just a specification of remote calls exposed to the API consumers.
\paragraph{2}
Constructor Overloading : Constructors, used to create instances of an object, may also be overloaded in some object oriented programming languages. Because in many languages the constructor's name is predetermined by the name of the class, it would seem that there can be only one constructor. Whenever multiple constructors are needed, they are implemented as overloaded functions. A default constructor takes no parameters, instantiating the object members with a value zero.
\paragraph{3}
Abstraction : Abstraction is a process by which concepts are derived from the usage and classification of literal ("real" or "concrete") concepts, first principles, or other methods.
Abstractions may be formed by reducing the information content of a concept or an observable phenomenon, typically to retain only information which is relevant for a particular purpose.
\paragraph{4}
Modularization : Modularixation means dividing the application or problem into two or more simple problem and by solve them we solve main big problem.
\paragraph{5}
Garbage Collection : In computer science, garbage collection (GC) is a form of automatic memory management. The garbage collector, or just collector, attempts to reclaim garbage, or memory occupied by objects that are no longer in use by the program.
\section{Problem 5}
\textsc{NetBeans project has been included} 
\section{Problem 6}
\textsc{NetBeans project has been included} 
\section{Problem 7}
\textsc{.java file has been included}
\section{Problem 8}
in problem 3.32 if we change 1 method it is better because if we get error we sure that error occur in method which we change it and if we change 1 method less change need than we change all the class.
\newline 
\textsc{BlueJ project has been included}
\end{document}
